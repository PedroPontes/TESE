\section{Motivation}
\label{sec:int_motivation}

\par Heat transfer in fluid-solid interfaces, with fluid state change, is a common phenomena in the nature and technology. It is still a mystery, although getting smaller with time, what mechanisms transfer heat during boiling. Because there are countless applications in the industry where this phenomena occurs, there is a need to know what actually is happening so we can then improve these applications by more efficiently controlling them. \\
\par Surface heat removal using liquids is a complex and fast happening that often involves state change. So, to study this type of phenomena, we need high precision equipment with, not only high spatial precision, but also time precision. In this field of study, many types of measuring equipment have been used to quantify and qualify such phenomena. The use of a thermocouple, for instance, which is a really common method, can be intrusive to the measured process, can only measure one point and cannot be in contact with electricity. With this in mind, infra-red thermography has been a great alternative to some of the existing intrusive temperature measuring methods. A thermographical camera with a good calibration can give high precision temperature results at high frame rates, which can give high definition qualitatively and, more importantly, quantitatively accurate thermal images. The IR camera also outputs two dimensional images, a great advantage when trying to understand this kinds of processes. \\
\par Although the IR camera use will be centered in the boiling process, the heat transfer mechanisms in droplet surface impact will also be studied. Both require high precision results, so calibration and result processing should be studied with great care. \\
\par While this work is developed, a computational study by Emanuele Teodori is being made, and this work's results will also be used to validate the computational model in use.