\section{Objectives}
\label{sec:int_contributions}

\par The main objective of this dissertation is to optimize the use of the IR Camera to study interface phenomena in droplet impact. This not only includes improving studying the positioning of the camera and data processing, but also study the techniques that are involved in getting to the heat transfer in the various interfaces. The techniques used will be an adaptation of what previous authors did using thin foil surfaces, always trying to improve both the time and spatial resolution.\\

\par To achieve good quantitative results with the camera, it is of the most importance that this camera is properly calibrated, so another important objective will be create a process that can accurately calibrate a camera for the laboratory's use. To add to a good calibration it will be also very important to create quality IR data processing tools that can be used in future work.\\

\par One last objective will be to vary the experiments variables and analyze if the collected images react as expected to the variation. This setup will be tested with foils at under, during and over saturation temperature, with an elevated and lower droplet impact velocity, with different liquids and also different surface wettability. This will evaluate if the improvements to the existent method allowed better results.